\chapter{INTRODUCTION}

\section{General Introduction}

In recent years, autonomous systems have become increasingly prevalent in various domains, ranging from warehouse automation and agricultural robotics to healthcare logistics and biosecurity surveillance. The rapid advancement of these systems has been driven by their potential to improve operational efficiency, reduce human exposure to hazardous environments, and enable continuous monitoring capabilities that surpass human limitations. Grid-based pathfinding, which represents environments as discrete cells with associated traversability and cost information, has emerged as a fundamental component in the navigation and planning systems of autonomous robots and vehicles. This approach provides a simple yet powerful abstraction that unifies geometric representation with obstacle detection, making it widely adopted in both research and industrial applications.

Grid maps serve as a versatile representation framework that bridges the gap between raw sensor data and high-level planning algorithms. By discretizing continuous spaces into regular cells, grid maps enable efficient spatial reasoning, obstacle representation, and cost-based planning. This representation is particularly effective in structured environments such as warehouses, greenhouses, hospital wards, and agricultural facilities where rows, aisles, and corridors naturally align with the grid structure. Moreover, the grid-based approach extends seamlessly to specialized applications, including biosecurity contexts where each cell can encode not only traversability but also additional risk metrics such as contamination likelihood, exposure levels, or quarantine status. This multi-dimensional representation capability makes grid maps especially suitable for environments where safety considerations are paramount.

Despite their widespread adoption, grid-based pathfinding systems face several critical challenges that limit their effectiveness in real-world autonomous applications. The first major challenge is computational complexity, particularly on large-scale grids where the state space grows quadratically with the environment dimensions. Classical search algorithms such as Dijkstra's algorithm, Breadth-First Search (BFS), and A* can suffer from extensive node expansions, especially in cluttered environments with narrow passages or complex obstacle configurations. This computational burden becomes even more pronounced when planning must be performed repeatedly in response to dynamic environment changes.

The second challenge relates to the dynamic nature of real-world environments. Occupancy information frequently changes due to moving obstacles, temporary blockages, personnel movement, or updated sensor readings. In biosecurity applications, risk annotations may be updated based on new test results, temporary quarantines, or detected contamination zones. Traditional planning algorithms typically require complete re-planning when the environment changes, leading to significant computational overhead and delayed responses. This limitation is particularly problematic in time-critical scenarios where rapid adaptation to environmental changes is essential for mission success and safety.

The third challenge involves the exploitation of environmental structure. Many real-world environments exhibit inherent regularities and symmetries that could potentially be leveraged to accelerate planning. Structured facilities such as greenhouses, warehouses, hospital wards, and agricultural storage areas often feature symmetric layouts with parallel rows, corridors, or aisles. However, classical planning algorithms treat each state independently and fail to exploit these structural properties, resulting in redundant computation and missed opportunities for efficiency gains. The lack of symmetry-aware planning mechanisms represents a significant gap in current grid-based navigation systems.

In biosecurity contexts, an additional dimension of complexity arises from the need to minimize not only path length or travel time but also cumulative exposure to contaminated or high-risk areas. This risk-aware planning requirement transforms the pathfinding problem into a multi-objective optimization challenge where the autonomous system must balance conflicting goals of efficiency and safety. Traditional algorithms can incorporate risk as an additional cost term, but they do not provide specialized mechanisms to focus computational effort on finding paths that explicitly minimize exposure while maintaining acceptable path quality. Furthermore, the integration of risk-aware planning with real-time responsiveness requirements creates additional technical challenges that current systems struggle to address effectively.

The deployment of grid-based pathfinding systems in practical autonomous applications also faces a critical gap between algorithmic development and real-world execution. While numerous planning algorithms have been proposed and evaluated in simulation environments, there remains limited open-source infrastructure for reliably converting planned grid paths into executable vehicle commands. This deployment gap is particularly evident in biosecurity robotics, where safe and reproducible path execution is essential for mission-critical tasks such as surveillance transects, targeted sampling routes, and disinfection operations. The lack of accessible, end-to-end toolchains from planning to flight execution represents a significant barrier to the practical adoption of advanced pathfinding techniques.

This thesis addresses these challenges through the development and evaluation of novel grid-based pathfinding techniques that combine computational efficiency with practical deployability. The research focuses on two complementary algorithmic innovations: Incremental Line Search (ILS), a corridor-constrained planning approach that reduces exploration while preserving path quality, and Folding A*, a symmetry-aware planning method that exploits horizontal symmetry to achieve constant-factor state-space reductions. Both techniques are designed to be compatible with risk-annotated grids, enabling their application in biosecurity contexts where exposure minimization is critical. Additionally, the thesis presents an open-source planning-to-flight pipeline that integrates these algorithms with ArduPilot's Software-In-The-Loop (SITL) simulator, demonstrating practical path execution capabilities for autonomous vehicles.

The motivation for this research stems from the growing demand for responsive, efficient, and safe autonomous navigation in structured environments, particularly in biosecurity applications where rapid response and exposure minimization can significantly impact containment effectiveness and human safety. By developing algorithms that exploit environmental structure and constrain search space while maintaining optimality guarantees, this thesis aims to bridge the gap between theoretical planning algorithms and practical autonomous systems deployment. The proposed techniques are evaluated comprehensively using both synthetic and realistic grid scenarios, including risk-annotated environments that reflect quarantine zones, contamination areas, and dynamic occupancy updates characteristic of real-world biosecurity operations.


\section{Problem Statement}

The deployment of autonomous systems for grid-based navigation faces several interconnected challenges that limit their effectiveness, efficiency, and practical deployability, particularly in time-critical and risk-sensitive applications. This section identifies four key problems that motivate the research objectives and methodological approaches of this thesis.

\textbf{Problem 1 (P1): Real-time Feasibility.} Standard pathfinding algorithms, including A*, Dijkstra's algorithm, and Breadth-First Search, can exhibit prohibitively long computation times on large-scale or highly cluttered grid maps, particularly when risk annotations are incorporated into the planning process. As grid dimensions increase, the state space grows quadratically, leading to extensive node expansions that consume significant computational resources. In risk-annotated grids commonly used in biosecurity applications, where each cell contains both traversability and exposure information, the planning process must evaluate composite cost functions that consider both distance and risk metrics. This additional complexity exacerbates the computational burden, making it challenging to meet real-time requirements on commodity hardware. The problem is further compounded in cluttered environments with narrow aisles or complex obstacle configurations, where search algorithms must explore numerous alternative paths before identifying an acceptable solution. For time-critical missions such as biosecurity surveillance, rapid response to detected threats, or emergency disinfection operations, the inability to compute feasible paths within strict time constraints can compromise mission effectiveness and potentially endanger human operators or critical assets.

\textbf{Problem 2 (P2): Responsiveness to Dynamic Updates.} Real-world environments are inherently dynamic, with occupancy and risk information frequently changing due to various factors including moving obstacles, personnel activity, sensor updates, test results, and evolving contamination zones. In biosecurity contexts, risk maps may be updated based on new pathogen detection, revised quarantine boundaries, or temporal progression of contamination spread. Traditional planning approaches typically respond to such changes by performing complete re-planning from scratch, discarding all previously computed information and restarting the search process. This complete re-planning strategy incurs substantial computational overhead, especially when updates occur frequently or when the environment is large and complex. The resulting delays in plan generation can be critical in time-sensitive scenarios where rapid adaptation is essential for maintaining safety margins and mission effectiveness. Moreover, frequent re-planning can lead to inconsistent or oscillating path selections, potentially confusing operators or compromising the reliability of autonomous operations. The lack of incremental planning mechanisms that can efficiently update existing plans in response to localized environmental changes represents a significant limitation of current grid-based navigation systems.

\textbf{Problem 3 (P3): Deployment Gap.} Despite the extensive research on grid-based pathfinding algorithms and their evaluation in simulation environments, there exists a significant gap between algorithmic development and practical deployment on physical autonomous vehicles. This deployment gap manifests in several ways. First, there is limited availability of open-source, end-to-end toolchains that integrate grid-based planning algorithms with vehicle control systems, particularly for risk-aware navigation scenarios. Second, the conversion of abstract grid paths into executable vehicle setpoints requires careful consideration of vehicle dynamics, control constraints, and safety margins, aspects that are often overlooked in pure algorithmic research. Third, the validation and testing of integrated planning-execution systems typically require access to specialized hardware and expertise that may not be readily available to researchers or practitioners. In biosecurity robotics, where safe and reproducible execution is paramount for tasks such as surveillance transects, sampling route navigation, and targeted disinfection, the lack of accessible deployment infrastructure represents a critical barrier to technology transfer and practical adoption. The absence of validated, open-source pipelines that demonstrate reliable path execution in simulation environments (such as Software-In-The-Loop systems) further impedes the transition from research prototypes to operational systems.

\textbf{Problem 4 (P4): Unexploited Environmental Symmetry.} Many practical autonomous navigation scenarios occur in structured environments that exhibit inherent symmetries, particularly horizontal symmetries in facilities such as greenhouses, warehouses, hospital wards, agricultural barns, and storage facilities. These environments typically feature parallel rows, corridors, or aisles that create symmetric patterns in the underlying grid representation. However, classical pathfinding algorithms treat each grid cell as an independent state and fail to recognize or exploit these structural regularities. This limitation results in redundant computation, as the algorithm may independently explore symmetric regions that could be processed collectively. The failure to exploit symmetry represents a missed opportunity for significant computational savings, as symmetric structures could enable constant-factor reductions in the effective state space size. Moreover, in risk-annotated grids where symmetry extends to both geometric layout and risk distribution, symmetry-aware planning could yield even greater benefits by simultaneously reducing computational cost and ensuring consistent risk-cost evaluation across symmetric regions. The absence of symmetry-exploitation mechanisms in existing grid-based pathfinding systems represents a fundamental inefficiency, particularly problematic in large-scale structured environments where symmetry is prevalent and predictable.

These four problems are interconnected and collectively highlight the need for novel algorithmic approaches that can simultaneously address computational efficiency, dynamic responsiveness, practical deployability, and structural exploitation in grid-based autonomous navigation. The solutions developed in this thesis specifically target these problems through complementary innovations in corridor-constrained search, symmetry-aware planning, and integrated planning-execution infrastructure.


\section{Research Questions and Research Hypotheses}

Based on the problems identified in Section 1.2, this research formulates four research questions and corresponding research hypotheses that guide the development, implementation, and evaluation of the proposed pathfinding techniques.

\textbf{Research Question 1 (RQ1):} To what extent can constraining search to an Incremental Line Search (ILS) corridor reduce computational requirements and cumulative exposure while preserving path optimality on risk-annotated grid maps?

\textbf{Research Hypothesis 1 (RH1):} The ILS approach will achieve significant reductions in runtime and node expansions (expected reduction of 40-70\% compared to standard A*) while maintaining path lengths within 5\% of optimal solutions. Additionally, on risk-annotated grids, ILS will reduce the cumulative exposure integral (sum of risk values along the path) by prioritizing paths through the corridor that naturally avoid high-risk regions, with negligible impact on path quality as measured by standard optimality metrics.

\textbf{Research Question 2 (RQ2):} Can an adaptive corridor mechanism that dynamically adjusts ILS corridor width based on local obstructions and risk concentrations support efficient re-planning under dynamic environmental updates while maintaining exposure constraints?

\textbf{Research Hypothesis 2 (RH2):} An adaptive corridor widening strategy that expands the search space only near detected obstructions or risk spikes will sustain the computational advantages of ILS (maintaining 50-80\% of the speedup observed in static scenarios) while successfully handling moderate environment dynamics. The adaptive mechanism will keep cumulative exposure within user-specified thresholds (e.g., no more than 15\% increase in exposure integral) even under dynamic updates, and will achieve re-planning latencies that are 3-5 times faster than complete re-planning approaches.

\textbf{Research Question 3 (RQ3):} Does an open-source planning-to-flight pipeline integrating ILS with risk-aware cost functions and ArduPilot Software-In-The-Loop (SITL) simulation reliably execute grid-derived paths for autonomous vehicles?

\textbf{Research Hypothesis 3 (RH3):} The integrated planning-to-flight pipeline will successfully execute planned paths in ArduPilot SITL with bounded tracking errors (position error $<$ 2 meters for waypoint following) and successful task completion rates exceeding 95\% for typical biosecurity mission profiles including surveillance transects, sampling routes, and disinfection patterns. The pipeline will demonstrate reproducible execution across multiple simulation runs with consistent performance metrics, providing a validated foundation for safe hardware deployment when permitted by safety protocols and regulatory requirements.

\textbf{Research Question 4 (RQ4):} When horizontal symmetry exists in grid environments, does the Folding A* algorithm provide predictable constant-factor acceleration while preserving optimality and correctly handling risk-weighted costs?

\textbf{Research Hypothesis 4 (RH4):} On horizontally symmetric grids, Folding A* will achieve approximately 50\% reduction in the effective state space (i.e., exploring roughly half the nodes compared to standard A*) with exact optimality preservation for both standard and risk-weighted cost functions. The runtime improvements will scale consistently across different grid sizes and obstacle densities, yielding speedups ranging from 1.8x to 2.2x compared to standard A* on symmetric maps. The algorithm will correctly handle risk-annotated grids by maintaining symmetry-aware risk propagation, ensuring that paths in symmetric regions are evaluated with equivalent composite costs.

These research questions and hypotheses are designed to be empirically testable through systematic experimentation using both synthetic grid scenarios and realistic environment models. The evaluation framework, detailed in Chapter 6, provides comprehensive metrics and experimental protocols to validate each hypothesis and answer the corresponding research question with quantitative evidence.


\section{Research Objectives}

The overarching aim of this research is to develop, implement, and validate efficient grid-based pathfinding techniques for autonomous systems that combine computational efficiency with practical deployability, particularly in risk-sensitive biosecurity applications. This aim is achieved through four specific research objectives:

\textbf{Objective 1 (O1):} Design and evaluate an Incremental Line Search (ILS) framework for risk-aware pathfinding that focuses exploration within a narrow corridor centered on a direct line between start and goal positions. The framework must integrate seamlessly with standard grid planning algorithms (A*, Dijkstra, BFS), support optional risk-weighted cost functions, and demonstrate measurable reductions in runtime and node expansions while maintaining near-optimal path quality. The evaluation will encompass diverse grid scenarios including varying sizes (50x50 to 1000x1000), obstacle densities (10\%-40\%), and risk annotation patterns reflecting realistic biosecurity scenarios such as quarantine zones and contamination gradients.

\textbf{Objective 2 (O2):} Develop and validate an adaptive corridor control mechanism for ILS that dynamically adjusts corridor width based on local environmental features including obstacle concentrations and risk-level spikes. The adaptive mechanism must balance exploration efficiency with solution completeness, ensuring that narrow corridors are used in open regions while sufficient expansion occurs near obstructions or high-risk zones. The control strategy will be evaluated under dynamic environments with piecewise-static updates at various frequencies (1-10 updates per planning session), assessing its ability to maintain computational efficiency while preserving path quality and exposure constraints across update cycles.

\textbf{Objective 3 (O3):} Implement and formally validate the Folding A* algorithm for horizontally symmetric grid maps, providing rigorous proofs of correctness and optimality preservation for both standard and risk-weighted cost functions. The implementation must include automatic symmetry detection mechanisms, efficient folding transformations that halve the effective state space, and correct unfolding procedures to recover full-space paths. The empirical evaluation will quantify constant-factor speedup gains across diverse symmetric grid scenarios, analyzing how performance scales with grid size, obstacle density, and the degree of symmetry in the environment.

\textbf{Objective 4 (O4):} Build and release an open-source, end-to-end grid-planning-to-flight pipeline that integrates ILS and Folding A* algorithms with ArduPilot Software-In-The-Loop (SITL) simulation for autonomous vehicle execution. The pipeline must support risk-annotated grid inputs, automatic conversion of grid paths to vehicle waypoints with appropriate safety margins, and comprehensive logging for performance analysis. The implementation will be validated through systematic SITL experiments demonstrating successful execution of representative biosecurity mission profiles including surveillance transects, area coverage patterns, and targeted sampling or disinfection routes. The pipeline will be released under a permissive open-source license to facilitate community adoption and extension.


\section{Scope and Limitations}

This research focuses on grid-based pathfinding for autonomous navigation with specific scope boundaries and acknowledged limitations that define the applicability and generalizability of the results.

\textbf{Environment Representation:} The research considers two-dimensional occupancy grids with optional risk layers, representing environments as regular rectangular grids with specified resolutions. Grid cells are classified as either traversable or occupied, with traversable cells optionally annotated with continuous risk values normalized to the range [0, 1]. The evaluation encompasses grid sizes ranging from 50x50 to 1000x1000 cells and obstacle densities from 10\% to 40\% of the total grid area. Risk distributions include uniform random patterns, localized hotspots, and structured zones reflecting quarantine areas or contamination gradients. While two-dimensional grids are sufficient for many practical applications including ground vehicles and fixed-altitude aerial vehicles, full three-dimensional environments with varying altitude constraints are considered out of scope for this work.

\textbf{Agent Model and Motion Primitives:} The research assumes a single autonomous agent with either 4-connected (cardinal directions) or 8-connected (including diagonals) motion primitives. For aerial vehicle scenarios, a fixed-altitude abstraction is employed, effectively reducing the problem to two-dimensional planning. The agent is modeled as occupying a single grid cell at any discrete time step, with collision checking performed at the cell level. Vehicle dynamics, differential constraints, and continuous-time trajectory optimization are addressed at the execution layer (ArduPilot SITL) rather than in the planning phase. Multi-agent coordination, formation control, and cooperative planning involving multiple simultaneous vehicles are explicitly excluded from the current scope and represent opportunities for future research extensions.

\textbf{Environment Dynamics:} The research considers moderate, piecewise-static environment dynamics where occupancy and risk information updates occur at discrete time points with sufficient intervals for the planner to complete its computation. The evaluation includes scenarios with 1 to 10 updates per planning session, reflecting realistic operational conditions such as moving obstacles, temporary blockages, updated test results, or revised quarantine boundaries. Highly adversarial dynamics, worst-case update patterns designed to maximally challenge the planner, and continuous-time stochastic environment evolution are considered out of scope. The assumption of piecewise-static updates is justified by the observation that most practical biosecurity and autonomous navigation scenarios exhibit update frequencies that are slower than typical planning latencies on modern processors.

\textbf{Planning Objectives and Cost Functions:} The primary planning objective is to find collision-free paths from a specified start cell to a specified goal cell, subject to real-time latency constraints. For standard grids, path cost is defined as cumulative distance or travel time based on Euclidean or Manhattan metrics with uniform edge costs unless otherwise stated. For risk-annotated grids, a composite cost function is employed that combines distance with a weighted risk term: $cost = distance + \lambda \cdot \sum risk$, where $\lambda$ is a user-specified weight parameter balancing efficiency and safety objectives. The research evaluates multiple $\lambda$ values (0.1, 0.5, 1.0, 5.0) to assess sensitivity to risk-cost tradeoff preferences. The goal is to find near-optimal or optimal paths under the specified cost function rather than to discover Pareto-optimal solution sets balancing multiple competing objectives.

\textbf{Evaluation Metrics:} Performance assessment employs a comprehensive set of metrics including: runtime (wall-clock computation time), nodes expanded (number of states explored during search), path length (total distance in grid units), exposure integral (sum of risk values for all cells along the path), maximum on-path risk (highest risk value encountered), and re-plan latency (time required to update a plan following an environment change). Optional supplementary metrics include memory consumption and success rate under dynamic updates. All experiments are conducted on controlled hardware configurations (Intel Core i7 or equivalent, 16GB RAM) to ensure reproducibility and fair comparisons across algorithms.

\textbf{Input Assumptions:} The research assumes that occupancy information (obstacle locations) and optional risk annotations are provided as exogenous inputs from external perception or sensor systems. The source of this information, sensor fusion techniques, uncertainty quantification, and noise filtering are not addressed in this work. Edge costs are assumed uniform in both spatial dimensions unless explicitly varied for experimental purposes. Start and goal positions are assumed known and specified in advance, with replanning triggered by environment changes rather than goal updates. Scenarios with unknown goals, exploration objectives, or information-gathering tasks are outside the current scope.

\textbf{Explicitly Out-of-Scope Topics:} The following topics are explicitly excluded from this research: full three-dimensional trajectory generation with altitude variation, multi-agent path planning and coordination, hardware-level control system tuning beyond basic setpoint tracking, detailed vehicle dynamics modeling or continuous-time trajectory optimization, probabilistic or belief-space planning under uncertainty, active sensing or information-theoretic exploration strategies, and learning-based approaches that adapt pathfinding behavior based on experience. These exclusions are made to maintain focused research objectives and manageable scope while acknowledging their importance as directions for future work.

\textbf{Limitations:} Several limitations should be acknowledged regarding the generalizability and applicability of the proposed techniques. First, the computational gains of corridor-constrained ILS may diminish on highly tortuous maps where the direct line between start and goal passes through multiple large obstacles, requiring extensive corridor widening. Second, Folding A* provides benefits only when horizontal symmetry is present and detectable; environments with irregular layouts or asymmetric features will not benefit from this technique. Third, risk-map uncertainty modeling, including probabilistic contamination spread or temporal evolution of risk distributions, is not explicitly addressed; risk values are treated as deterministic inputs. Fourth, the ArduPilot SITL integration provides validation in simulation but does not replace the need for comprehensive hardware testing in controlled outdoor environments before operational deployment, which requires additional safety protocols and regulatory compliance beyond the scope of this thesis.


\section{Significance of the Research}

This research makes significant contributions to both theoretical understanding and practical deployment of autonomous navigation systems, with particular relevance to biosecurity applications where efficient, responsive, and risk-aware pathfinding can directly impact operational safety and mission effectiveness.

\textbf{Computational Efficiency for Real-Time Systems:} The proposed corridor-constrained Incremental Line Search (ILS) approach addresses a fundamental limitation of classical grid-based planners by dramatically reducing the search space that must be explored to find high-quality paths. By constraining exploration to a narrow corridor around the direct line between start and goal, ILS enables real-time pathfinding on large-scale grids using commodity hardware, without requiring specialized processors or GPU acceleration. This computational efficiency translates directly to improved system responsiveness, allowing autonomous vehicles to rapidly generate and update plans in dynamic environments. For time-critical applications such as emergency response, surveillance of sudden contamination events, or rapid deployment scenarios, the ability to compute paths within strict latency constraints (e.g., $<$ 1 second for 500x500 grids) can significantly enhance operational capability and mission success rates.

\textbf{Enhanced Safety Through Risk-Aware Planning:} The integration of risk-annotated grids with ILS and Folding A* enables autonomous systems to balance competing objectives of efficiency and safety in contaminated or hazardous environments. In biosecurity contexts, minimizing cumulative exposure to pathogen-laden areas can reduce infection probability for both autonomous agents (reducing decontamination requirements) and human operators (reducing occupational exposure risks). The explicit treatment of exposure integrals (sum of risk along the path) and maximum on-path risk provides quantifiable safety metrics that can be incorporated into mission planning and risk assessment workflows. By reducing average exposure while maintaining acceptable path lengths, the proposed techniques enable more aggressive deployment of autonomous systems in scenarios where human access would pose unacceptable safety risks, such as active outbreaks, quarantine zones, or contaminated agricultural facilities.

\textbf{Improved Responsiveness to Environmental Dynamics:} The adaptive corridor mechanism developed in Objective 2 specifically addresses the challenge of efficient re-planning under dynamic environment changes. By enabling localized corridor widening only where necessary to accommodate new obstacles or updated risk information, the adaptive approach maintains the computational efficiency of ILS while ensuring solution completeness under moderate dynamics. This capability is particularly valuable in biosecurity operations where risk maps are frequently updated based on new test results, revised contamination boundaries, or temporal disease spread models. The ability to rapidly adapt paths in response to these updates, with re-planning latencies significantly lower than complete search approaches, enables more responsive and adaptive autonomous operations that can track evolving threats and optimize mission execution in real time.

\textbf{Predictable Performance Through Symmetry Exploitation:} Folding A* makes a fundamental theoretical contribution by demonstrating how horizontal symmetry in grid environments can be exploited to achieve guaranteed constant-factor state-space reductions while preserving optimality. Unlike heuristic acceleration techniques that may provide variable speedups depending on problem characteristics, the symmetry-based reduction is predictable and analyzable, making it suitable for systems with hard real-time constraints or safety-critical requirements. The formal correctness and optimality proofs provided in Chapter 4 establish a rigorous foundation for applying Folding A* in operational systems, giving practitioners confidence that the performance gains do not come at the cost of solution quality degradation or incorrect risk-cost evaluation. For structured environments such as greenhouses, warehouses, and hospital wards where symmetry is common and persistent, Folding A* provides reliable, consistent acceleration that can be factored into system design and performance specifications.

\textbf{Practical Deployment Infrastructure:} The open-source planning-to-flight pipeline developed in Objective 4 addresses a critical gap between algorithmic research and practical deployment by providing a validated, reproducible pathway from grid-based planning to vehicle execution. The integration with ArduPilot SITL enables researchers and practitioners to evaluate and refine pathfinding algorithms in a realistic simulation environment that accurately models vehicle dynamics, control loops, and environmental interactions. This infrastructure reduces the barrier to entry for deploying advanced pathfinding techniques on physical vehicles, potentially accelerating technology transfer from research prototypes to operational systems. In biosecurity robotics, where safe and reliable execution is paramount, the availability of a tested SITL-based pipeline provides a crucial stepping stone toward hardware deployment, enabling extensive simulation-based validation before committing to field trials.

\textbf{Contribution to Biosecurity Robotics:} The specific focus on biosecurity applications represents a timely and important contribution given the increasing recognition of autonomous systems' potential role in disease surveillance, containment, and response. The techniques developed in this thesis directly support several critical biosecurity mission types including surveillance transects for early detection of contamination, targeted sampling route optimization to maximize information gain while minimizing exposure, and systematic disinfection coverage planning to ensure thorough treatment of affected areas. By enabling faster, safer, and more responsive autonomous operations, this research contributes to broader efforts to enhance biosecurity infrastructure and reduce dependence on human personnel in hazardous environments. The explicit treatment of exposure minimization, rapid re-planning capabilities, and validated execution pipelines collectively advance the state of practice in biosecurity robotics.

\textbf{Broader Impact on Autonomous Navigation:} While biosecurity applications provide the primary motivation and evaluation context, the techniques developed in this thesis have broader applicability to general autonomous navigation problems wherever grid-based planning is employed. The ILS corridor-constrained approach can benefit warehouse automation, agricultural robots navigating crop rows, search and rescue operations in structured environments, and any scenario where rapid pathfinding on large grids is required. Similarly, Folding A* can accelerate planning in any symmetric environment, from parking garages to indoor navigation in buildings with regular floor plans. The open-source release of implementations and the planning-to-flight pipeline facilitates adoption by researchers and practitioners across diverse application domains, potentially catalyzing innovation and accelerating the development of more capable autonomous systems.

In summary, this research advances the state of the art in grid-based pathfinding through novel algorithmic innovations (corridor-constrained search and symmetry exploitation), practical deployment infrastructure (planning-to-flight pipeline), and rigorous empirical validation in biosecurity-relevant scenarios. The combination of theoretical contributions (formal correctness proofs, complexity analysis) with practical validation (SITL experiments, real-world grid scenarios) ensures that the research findings are both scientifically rigorous and immediately applicable to operational autonomous systems.


\section{Organization of the Thesis}

This thesis is organized into seven chapters that systematically develop, implement, and evaluate the proposed pathfinding techniques for autonomous systems. The structure is designed to provide a logical progression from problem motivation through theoretical development, empirical validation, and concluding discussion.

\textbf{Chapter 1: Introduction} provides the foundational context for the research, including an overview of grid-based pathfinding in autonomous systems with particular emphasis on biosecurity applications, identification of key problems limiting current approaches, formulation of research questions and hypotheses, specification of research objectives, definition of scope and limitations, and articulation of the research significance. This chapter establishes the motivation and framework for the entire thesis.

\textbf{Chapter 2: Literature Review and Related Work} surveys the existing body of knowledge relevant to grid-based pathfinding and positions the contributions of this thesis within the research landscape. The chapter covers classical search algorithms (Dijkstra, BFS, A*) and their properties, heuristic search techniques and admissibility considerations, incremental and anytime planning approaches that handle dynamic environments, line-of-sight path shortcutting methods, symmetry exploitation techniques in pathfinding, and risk-sensitive or exposure-aware routing methods particularly in biosecurity robotics contexts. The review identifies gaps in current approaches that motivate the development of ILS and Folding A*, and establishes the novelty of the proposed techniques relative to prior art.

\textbf{Chapter 3: Incremental Line Search Framework} presents the detailed design, implementation, and analysis of the ILS corridor-constrained pathfinding approach. The chapter describes corridor construction algorithms based on direct line-of-sight between start and goal, adaptive widening strategies that expand the corridor near obstructions or risk concentrations, integration mechanisms with standard grid planners (A*, Dijkstra, BFS), theoretical complexity analysis comparing ILS to complete search, and benchmark results on risk-aware grid scenarios demonstrating runtime reductions, node expansion savings, and exposure integral improvements. This chapter addresses Research Questions 1 and 2 and fulfills Objectives 1 and 2.

\textbf{Chapter 4: Folding A* for Symmetric Grids} develops the symmetry-aware planning technique that exploits horizontal symmetry to achieve constant-factor state-space reductions. The chapter defines horizontal symmetry formally on grid maps, describes the folding transformation that projects the full grid onto a half-size representation, presents comprehensive correctness and optimality proofs including extensions to risk-weighted cost functions, analyzes theoretical complexity showing guaranteed 2x state-space reduction, and reports experimental results quantifying speedups across diverse symmetric grids with varying sizes and densities. This chapter addresses Research Question 4 and fulfills Objective 3.

\textbf{Chapter 5: Open-Source Planning-to-Flight Pipeline} documents the integrated system that bridges grid-based planning algorithms with vehicle execution using ArduPilot Software-In-The-Loop simulation. The chapter describes the pipeline architecture including grid input processing, ILS and Folding A* integration, path-to-waypoint conversion with safety margins, ArduPilot SITL interface and configuration, and comprehensive logging and performance analysis tools. The chapter presents experimental validation demonstrating successful execution of biosecurity mission profiles including surveillance transects, sampling routes, and disinfection patterns, with quantitative tracking error analysis and reproducibility assessment. This chapter addresses Research Question 3 and fulfills Objective 4.

\textbf{Chapter 6: Results and Discussion} provides comprehensive comparative evaluations and in-depth analysis of the proposed techniques across diverse experimental scenarios. The chapter presents comparative performance analysis of ILS versus standard A*/Dijkstra/BFS on grids of varying sizes and densities, evaluation of Folding A* speedups on symmetric grids with different symmetry characteristics, ablation studies isolating the impact of corridor width, adaptive widening parameters, and risk weighting factors, sensitivity analyses examining robustness to parameter variations and dynamic environment changes, and practical implications for biosecurity operations including exposure reduction quantification and mission time improvements. The chapter synthesizes results across all research questions, validates or refines the research hypotheses based on empirical evidence, and discusses practical considerations for deploying the techniques in operational systems.

\textbf{Chapter 7: Conclusions, Limitations, and Future Work} synthesizes the key findings of the thesis and outlines directions for continued research. The chapter summarizes the main contributions including algorithmic innovations, theoretical results, empirical findings, and practical deployment infrastructure, discusses limitations of the current work including assumptions that may not hold in all scenarios and cases where the proposed techniques may not provide advantages, and identifies promising avenues for future research such as extensions to three-dimensional environments, integration with learning-based approaches, applications to multi-agent coordination, and field deployment on physical hardware in operational biosecurity contexts.

This organizational structure ensures a coherent narrative flow from motivation through method development to validation and conclusion, while maintaining clear connections between theoretical innovations, implementation details, and empirical results throughout the thesis.
