\chapter{LITERATURE REVIEW}\label{chap:review}

This chapter reviews the existing body of knowledge on grid-based pathfinding for autonomous navigation, with particular emphasis on techniques applicable to risk-sensitive and dynamic environments such as port facilities. The review is organised thematically: Section~\ref{sec:problem_setting} establishes the problem setting and environment representations used in autonomous navigation research; Section~\ref{sec:heterogeneity} examines how spatial and temporal heterogeneity complicates pathfinding; Section~\ref{sec:classical_search} surveys classical graph search algorithms; Section~\ref{sec:replanning} reviews incremental and real-time replanning methods; Section~\ref{sec:symmetry} discusses symmetry reduction and search-space compression techniques; Section~\ref{sec:learning_hybrid} covers learning-based and hybrid planning approaches; Section~\ref{sec:evaluation} addresses optimisation, parameter tuning, and evaluation protocols; Section~\ref{sec:gaps} identifies research gaps motivating the present work; and Section~\ref{sec:review_summary} provides a summary.


\section{Problem Setting and Environment Representation}\label{sec:problem_setting}

\subsection{Port Operational Context}\label{subsec:port_context}

Modern container ports are complex cyber-physical environments where autonomous navigation must contend with dense, dynamic obstacle fields and strict operational constraints \citep{Notteboom2024, Kim2023Port}. Container stacking yards give rise to structured obstacle patterns characterised by parallel blocks separated by narrow aisles, while quay-side crane operations, truck movements, and personnel activity introduce temporal variability \citep{ENISA2023, Huang2024}. The challenge of balancing path efficiency against collision risk and operational disruption drives the study of risk-aware pathfinding techniques that can operate in real time under dynamic conditions \citep{Raja2024, Chung2022}.

Recent deployments of autonomous ground vehicles and unmanned aerial systems in port environments have highlighted both the promise and the shortcomings of current pathfinding technology \citep{Kim2023Port, CAAM2024}. Ground-based automated guided vehicles typically follow predefined routes, while UAV-based inspection and surveillance missions call for flexible pathfinding that adapts to changing conditions \citep{Huang2024, DroneRegulations2024}. The growing adoption of digital twin technology in port management provides high-fidelity environment models that can serve as inputs to grid-based planners \citep{Notteboom2024}.

\subsection{Map Models: Occupancy Grids and Cost Maps}\label{subsec:map_models}

Grid-based environment representations discretise continuous space into regular cells, each classified as traversable or occupied \citep{Russell2021, Das2023}. Uniform-cost occupancy grids assign identical traversal costs to all free cells, supporting algorithms that take advantage of this uniformity for computational efficiency \citep{Cormen2022}. Weighted-cost grids extend this model by associating each traversable cell with a continuous cost value reflecting terrain difficulty, risk exposure, or energy consumption \citep{Patel2023, Wang2024Risk}.

For risk-aware navigation, a composite cost function combines distance with weighted risk:
\begin{equation}\label{eq:composite_cost}
  \mathrm{cost}(n, n') = \mathrm{dist}(n, n') + \lambda \cdot r(n'),
\end{equation}
where $\mathrm{dist}(n, n')$ is the step distance, $r(n') \in [0,1]$ is the risk annotation of cell $n'$, and $\lambda \geq 0$ is a user-specified weight controlling the distance--risk tradeoff \citep{Wang2024Risk, Zheng2023Risk}. This formulation has been widely adopted in risk-sensitive path planning for UAVs and autonomous vehicles \citep{Phung2021, PuenteCastro2022}.

Hierarchical representations---quad-trees, navigation meshes, and multi-resolution grids---reduce computational requirements by adapting resolution to local complexity \citep{Zafar2023, Das2023}. However, these representations introduce approximation errors and complicate the application of theoretical guarantees that were originally derived for uniform grids \citep{Sturtevant2022}.

\subsection{Sensing and Dynamic Map Updates}\label{subsec:sensing}

Autonomous systems acquire environment information through onboard sensors (LiDAR, cameras, ultrasonic rangefinders) and external sources (satellite imagery, port management systems) \citep{Zhao2023Survey, Tsouros2023}. Sensor data are fused into occupancy grids through probabilistic frameworks that maintain confidence estimates for cell states \citep{Russell2021}. Dynamic map updates---arising from moving obstacles, revised risk assessments, or new operational boundaries---require planners capable of efficiently incorporating changes without full re-computation \citep{Andreychuk2022, Hernandez2021}.

Simulation platforms provide controlled environments for evaluating pathfinding algorithms under reproducible conditions. Table~\ref{tab:environment_benchmarks} summarises prominent benchmarking environments used in recent autonomous navigation research.

\begin{table}[htb!]
\centering
\caption{Environment Benchmarks for Autonomous Navigation Research}
\label{tab:environment_benchmarks}
\begin{tabular}{|p{3cm}|p{3cm}|p{4cm}|p{3cm}|}
\hline
\textbf{Environment} & \textbf{Reference} & \textbf{Key Features} & \textbf{Limitations} \\
\hline
AirSim & \citet{Shah2018} & High-fidelity visual simulation, Unreal Engine & Computational overhead \\
\hline
Moving AI Benchmarks & \citet{Sturtevant2022} & Standardised grid maps, game-derived scenarios & Static environments only \\
\hline
ArduPilot SITL & \citet{ArduPilot2024} & Vehicle dynamics, MAVLink integration & Limited visual fidelity \\
\hline
Gazebo Sim & \citet{GazeboSim2024} & Physics engine, ROS integration, multi-robot & Configuration complexity \\
\hline
\end{tabular}
\end{table}


\section{Heterogeneity and Dynamics in Pathfinding}\label{sec:heterogeneity}

\subsection{Spatial Heterogeneity}\label{subsec:spatial}

Real-world environments exhibit substantial spatial variation in traversability and risk \citep{Das2023, Wang2024Risk}. In port settings, open marshalling areas permit rapid traversal, while narrow inter-block aisles constrain movement to single-file corridors \citep{Kim2023Port, Notteboom2024}. Risk annotations may reflect proximity to hazardous cargo, contamination zones, or areas with heavy equipment traffic \citep{Huang2024, Zheng2023Risk}.

Container stacking configurations create structured obstacle patterns that differ qualitatively from random obstacle distributions \citep{Kim2023Port}. Containers are arranged in standardised blocks with regular spacing, creating parallel corridors oriented along primary axes. This structural regularity introduces symmetry properties that classical algorithms fail to exploit, instead treating symmetric regions as independent states requiring separate exploration \citep{Bartak2022, Shleyfman2021}.

Narrow passages and chokepoints require special attention in grid-based pathfinding \citep{Chen2022Search}. When feasible paths must pass through constrained corridors, search algorithms may expand large numbers of nodes in unsuccessful exploration of blocked alternatives before finding the viable route. Jump Point Search addresses this challenge for uniform-cost grids through aggressive pruning of intermediate nodes \citep{Hu2021Bidir, Harabor2014}, but its restriction to uniform costs precludes application to risk-weighted grids \citep{Patel2023, Xu2024Grid}.

\subsection{Temporal Variability}\label{subsec:temporal}

Port environments undergo continuous change as vessels arrive and depart, containers are loaded and unloaded, and operational zones are reconfigured \citep{Kim2023Port, Notteboom2024}. These dynamics manifest as changes to both obstacle configurations and risk distributions. A pathfinding system must detect relevant changes and update its plan within the latency bounds dictated by the vehicle's speed and the proximity of obstacles \citep{Andreychuk2022, Yakovlev2024}.

The frequency and magnitude of environmental changes determine which replanning strategies are most suitable \citep{Hernandez2021}. Infrequent, localised changes favour incremental approaches that repair existing solutions \citep{Nash2021}; frequent, widespread changes may demand complete re-planning from scratch \citep{Zhou2022Review}. Characterising the change regime is therefore essential for choosing the right pathfinding algorithm \citep{SternLi2021}.

\subsection{Spatio-Temporal Coupling}\label{subsec:coupling}

Spatial heterogeneity and temporal variability interact in complex ways \citep{Zhao2023Survey, PuenteCastro2022}. A narrow corridor that is currently traversable may become blocked by a moving crane, requiring both detection of the change and computation of an alternative route through a potentially distant passage. Risk annotations may evolve as contamination spreads or as new sensor readings update exposure estimates \citep{Zheng2023Risk, Wang2024Risk}.

This coupling calls for pathfinding frameworks that integrate spatial cost models with temporal update mechanisms. Algorithms must not only compute paths through heterogeneous cost landscapes but also adapt those paths efficiently when the landscape changes \citep{Andreychuk2022, Patel2023}. The interplay between corridor structure and dynamic obstacles in port environments makes this integration particularly challenging, since blocked corridors may require substantial path rerouting \citep{Chen2022Search, Huang2024}.


\section{Classical Graph Search for Grid-Based Navigation}\label{sec:classical_search}

\subsection{A* and Admissible Heuristics}\label{subsec:astar}

A* search \citep{Hart1968, Hart1972} is the foundational algorithm for optimal pathfinding on weighted graphs. A* maintains an open list of candidate nodes ordered by the evaluation function $f(n) = g(n) + h(n)$, where $g(n)$ is the cost from start to $n$ and $h(n)$ is a heuristic estimate of the cost from $n$ to the goal. When the heuristic is admissible---meaning $h(n) \leq h^{*}(n)$ for all nodes---A* is guaranteed to find an optimal path \citep{Russell2021, Cormen2022}.

On grid maps, common admissible heuristics include Manhattan distance for 4-connected grids and octile distance for 8-connected grids \citep{Sturtevant2022, Das2023}. These geometric heuristics remain admissible when risk-weighted costs are added, since the risk component only increases actual path costs beyond the geometric lower bound \citep{Patel2023}.

How well A* performs depends critically on the quality of the heuristic: a perfectly informed heuristic ($h = h^{*}$) causes A* to expand only nodes on optimal paths, while the trivial heuristic ($h = 0$) reduces A* to Dijkstra's algorithm \citep{Cormen2022, Kuroiwa2022}. In practice, grid pathfinding heuristics provide moderate guidance, and A* typically expands substantially fewer nodes than uninformed search while remaining more expensive than methods that exploit structural properties of the grid \citep{Husvogt2024, Xu2024Grid}.

\subsection{Dijkstra's Algorithm and Breadth-First Search}\label{subsec:dijkstra_bfs}

Dijkstra's algorithm \citep{Dijkstra1959} finds shortest paths from a source to all reachable vertices in a graph with non-negative edge weights, expanding vertices in order of increasing distance from the source. On a grid with $N$ cells and using a binary-heap priority queue, Dijkstra's algorithm runs in $O(N \log N)$ time \citep{Cormen2022}. Breadth-first search (BFS) is the special case for unweighted graphs, running in $O(N)$ time on grids with uniform traversal costs \citep{Russell2021}.

While A* dominates Dijkstra for single-pair shortest paths when a good heuristic is available, Dijkstra remains relevant for computing all-pairs shortest paths and as a baseline in experimental comparisons \citep{Sturtevant2022, SternLi2021}. Recent work has shown that Dijkstra-based preprocessing can accelerate online queries through precomputed distance tables or subgoal graphs \citep{Strasser2022, Uras2023}.

\subsection{Heuristic Design and Weighted Search}\label{subsec:heuristic_design}

Weighted A* inflates the heuristic by a factor $\epsilon > 1$, using $f(n) = g(n) + \epsilon \cdot h(n)$, trading optimality for speed by biasing search toward the goal \citep{Pohl1970, Wilt2012}. The resulting path cost is bounded by $\epsilon$ times the optimal cost, providing a tunable quality--speed tradeoff \citep{Russell2021}. Recent analyses have characterised scenarios where weighted A* fails to deliver meaningful speedup, underscoring the importance of heuristic quality and search-space topology \citep{Helmert2023, Kuroiwa2022}.

Any-angle pathfinding algorithms such as Theta* \citep{Daniel2010} and ANYA* \citep{Patel2023} compute paths that are not constrained to grid edges, producing shorter and more natural trajectories. These algorithms are especially relevant for vehicle navigation where grid-aligned paths introduce unnecessary turns \citep{Nash2021}. Recent advances have extended any-angle pathfinding to weighted grids, opening the door to risk-annotated environments \citep{Patel2023, Xu2024Grid}.

Table~\ref{tab:classical_methods} summarises classical search methods and their properties, ordered chronologically by reference.

\begin{table}[htb!]
\centering
\caption{Classical Search Methods for Grid-Based Pathfinding}
\label{tab:classical_methods}
\begin{tabular}{|p{2.5cm}|p{2.5cm}|p{4cm}|p{3.5cm}|}
\hline
\textbf{Method} & \textbf{Reference} & \textbf{Strengths} & \textbf{Limitations} \\
\hline
Theta* & \citet{Daniel2010} & Any-angle paths, shorter trajectories & Line-of-sight checks costly \\
\hline
Weighted A* & \citet{Wilt2012} & Bounded suboptimality, tunable speed & Quality degrades with $\epsilon$ \\
\hline
A* & \citet{Russell2021} & Optimal, complete, well-understood & Slow on large grids \\
\hline
JPS & \citet{Hu2021Bidir} & Aggressive pruning, fast on uniform grids & Uniform costs only \\
\hline
BFS / Dijkstra & \citet{Cormen2022} & Complete, simple implementation & No heuristic guidance \\
\hline
Improved A* & \citet{Zhou2022Review} & Dynamic obstacle handling & Problem-specific tuning \\
\hline
ANYA* & \citet{Patel2023} & Optimal any-angle, weighted grids & Complex implementation \\
\hline
Bidirectional A* & \citet{Husvogt2024} & Reduced search space & Meeting-point heuristic design \\
\hline
Weighted grid search & \citet{Xu2024Grid} & Efficient on heterogeneous costs & Preprocessing overhead \\
\hline
\end{tabular}
\end{table}


\section{Incremental and Real-Time Replanning Methods}\label{sec:replanning}

Dynamic environments require pathfinding algorithms that can update solutions efficiently when the environment changes. This section reviews incremental search methods that reuse computation across successive planning episodes, as well as real-time approaches that interleave planning with execution \citep{SternLi2021, Hernandez2021}.

\subsection{D*, D* Lite, and LPA*}\label{subsec:dstar}

The D* family of algorithms \citep{Stentz1994, Stentz1995} pioneered incremental replanning for robot navigation in partially known environments. D* Lite \citep{Koenig2002} simplified the original D* algorithm while preserving equivalent functionality, using backward search from goal to start to efficiently handle cost changes encountered during forward traversal. Lifelong Planning A* (LPA*) \citep{Koenig2004} introduced the rhs-value mechanism for detecting locally inconsistent nodes, enabling selective re-expansion of only those nodes affected by cost changes.

These algorithms maintain data structures across planning episodes, avoiding redundant computation when changes are localised. However, their space complexity remains $O(N^2)$ for an $N \times N$ grid, since the entire graph must be stored to support incremental updates \citep{Andreychuk2022}. Recent work has extended incremental search to continuous-time domains \citep{Andreychuk2022} and multi-agent settings \citep{Stern2024, Li2024}.

Anytime Repairing A* (ARA*) \citep{Likhachev2003} takes a complementary approach by computing an initial suboptimal solution quickly and then progressively improving it as time permits. Anytime D* (AD*) \citep{Likhachev2008} combines anytime properties with incremental repair, supporting both quality improvement and dynamic response. Truncated Incremental Search \citep{Aine2016} cuts the overhead of incremental methods by limiting re-expansion to nodes within a bounded region of the changed edges, achieving practical speedups on large grids.

\subsection{Dynamic Obstacle Handling}\label{subsec:dynamic_obstacles}

Real-time obstacle detection and response require tight integration between perception and planning subsystems \citep{Zhao2023Survey, Raja2024}. When a previously unknown obstacle is detected, the planner must update the affected grid cells and compute an alternative path before the vehicle reaches the obstructed region. The latency budget for this update depends on vehicle speed, obstacle proximity, and safety margins \citep{Yakovlev2024, Wang2024Risk}.

Safe Interval Path Planning (SIPP) handles dynamic obstacles by computing paths that avoid time--space conflicts \citep{Andreychuk2022}. Rather than treating obstacles as permanently blocked cells, SIPP identifies safe time intervals during which each cell is traversable, allowing the planner to route through cells that are only temporarily occupied. Recent extensions support anytime solutions and multi-agent coordination \citep{Yakovlev2024, Stern2024}.

Table~\ref{tab:replanning_methods} summarises incremental and replanning methods.

\begin{table}[htb!]
\centering
\caption{Incremental and Replanning Methods}
\label{tab:replanning_methods}
\begin{tabular}{|p{2.5cm}|p{2.5cm}|p{4cm}|p{3.5cm}|}
\hline
\textbf{Method} & \textbf{Reference} & \textbf{Strengths} & \textbf{Limitations} \\
\hline
D* Lite & \citet{Koenig2002} & Efficient incremental repair & Full-grid memory \\
\hline
ARA* & \citet{Likhachev2003} & Anytime, bounded suboptimality & No incremental repair \\
\hline
LPA* & \citet{Koenig2004} & Selective re-expansion & Full-grid memory \\
\hline
AD* & \citet{Likhachev2008} & Anytime + incremental & Complex implementation \\
\hline
Truncated IS & \citet{Aine2016} & Bounded re-expansion scope & Approximation tradeoff \\
\hline
Incremental Phi* & \citet{Nash2021} & Any-angle + incremental & Implementation complexity \\
\hline
SIPP & \citet{Andreychuk2022} & Time-dependent planning & Requires temporal model \\
\hline
Anytime SIPP & \citet{Yakovlev2024} & Anytime + safe intervals & Computational overhead \\
\hline
\end{tabular}
\end{table}


\section{Symmetry Reduction and Search-Space Compression}\label{sec:symmetry}

\subsection{Motivation: Redundancy in Grid Search}\label{subsec:symmetry_motivation}

Grid-based search algorithms suffer from substantial redundancy when the environment contains structural regularities \citep{Shleyfman2021, Bartak2022}. On a uniform-cost grid, many distinct grid paths between two points have identical costs, differing only in the order of cardinal and diagonal steps. This path symmetry causes A* to expand large numbers of equivalent nodes without gaining useful information, inflating computation well beyond the minimum needed to establish optimality \citep{Hu2021Bidir, Sturtevant2022}.

Jump Point Search (JPS) \citep{Harabor2014} tackled path symmetry on uniform-cost grids by identifying \emph{jump points}---nodes where the optimal path must change direction---and pruning all intermediate nodes that can be inferred from the jump point structure. JPS achieves order-of-magnitude speedups over A* on uniform grids by eliminating symmetric path expansions \citep{Hu2021Bidir, Sturtevant2020GPPC}. However, JPS requires uniform edge costs, which rules out direct application to weighted or risk-annotated grids \citep{Patel2023, Xu2024Grid}.

\subsection{Geometric Symmetry in Structured Environments}\label{subsec:geometric_symmetry}

Beyond path symmetry, many practical environments exhibit \emph{geometric symmetry}---reflection or rotational invariance in the spatial layout of obstacles and costs \citep{Shleyfman2021, Bartak2022}. Warehouses, greenhouses, hospital wards, and container yards frequently feature mirror-symmetric floor plans, with parallel rows or corridors creating bilateral symmetry about a central axis \citep{Kim2023Port, Raja2024}.

Symmetry exploitation in planning has been studied extensively in the context of state-space search and constraint satisfaction \citep{Shleyfman2021}. Techniques include symmetry-based pruning, which detects and removes symmetric states during search \citep{Bartak2022}, and symmetry-based abstraction, which collapses symmetric regions into representative states \citep{Helmert2023}. In multi-agent pathfinding, symmetry breaking prevents agents from exploring equivalent orderings \citep{Bartak2022, Li2024}.

However, the application of geometric symmetry to single-agent grid-based pathfinding remains limited. Existing approaches focus on path symmetry (equivalent orderings of grid steps) rather than geometric symmetry (invariance of the obstacle and cost structure under spatial transformations) \citep{Hu2021Bidir, Bono2019}. For weighted grids, where JPS does not apply, no established method leverages the geometric regularity of the environment to shrink the effective search space while preserving cost-optimal solutions \citep{Xu2024Grid, Patel2023}.

\subsection{Subgoal Graphs and Spatial Decomposition}\label{subsec:subgoal}

Subgoal graph methods reduce the search space by identifying a sparse set of \emph{subgoal} nodes sufficient to represent all optimal paths \citep{Strasser2022, Uras2023}. Preprocessing identifies subgoals at obstacle corners and other structurally significant locations, then constructs a graph connecting subgoals with precomputed shortest-path distances. Online queries search only the subgoal graph, achieving substantial speedups at the cost of preprocessing time and memory \citep{Sturtevant2020GPPC}.

Recent advances in subgoal-based methods have improved both preprocessing efficiency and query performance \citep{Strasser2022, Uras2023}. The Grid-Based Path Planning Competition (GPPC) has served as a standardised evaluation platform, driving algorithmic improvements through competitive benchmarking \citep{Sturtevant2022, Sturtevant2020GPPC}. However, subgoal methods require that the environment remains static during the preprocessing phase; dynamic environments invalidate precomputed subgoal graphs and call for costly re-preprocessing \citep{Hernandez2021}.

Table~\ref{tab:symmetry_methods} summarises symmetry and search-space reduction methods.

\begin{table}[htb!]
\centering
\caption{Symmetry Reduction and Search-Space Compression Methods}
\label{tab:symmetry_methods}
\begin{tabular}{|p{3cm}|p{2.5cm}|p{4cm}|p{3cm}|}
\hline
\textbf{Method} & \textbf{Reference} & \textbf{Approach} & \textbf{Limitations} \\
\hline
Rectangular symmetry & \citet{Harabor2010} & Identify symmetric rectangular regions & Uniform costs only \\
\hline
JPS & \citet{Harabor2014} & Jump point pruning of symmetric paths & Uniform costs only \\
\hline
Symmetry in planning & \citet{Shleyfman2021} & Symmetry-based state pruning & General planning focus \\
\hline
Probabilistic symmetry & \citet{Bono2019} & Probabilistic symmetry detection & Multi-agent focus \\
\hline
Symmetry breaking MAPF & \citet{Bartak2022} & Break equivalent agent orderings & Multi-agent focus \\
\hline
Subgoal graphs & \citet{Strasser2022} & Sparse subgoal representation & Static environments \\
\hline
\end{tabular}
\end{table}

% --- Transition from Symmetry to Learning-Based ---

The symmetry reduction and compression techniques reviewed above exploit \emph{structural} properties of the search space---regular obstacle patterns, path equivalences, and sparse subgoal representations---to prune redundant exploration. These methods work well when the relevant structure can be characterised analytically and detected efficiently. However, many environments exhibit regularities that are hard to formalise as explicit symmetries or subgoal conditions, yet are readily apparent from data. This observation has spurred a complementary line of research: learning-based and hybrid planning methods that discover exploitable structure from experience rather than from geometric analysis.


\section{Learning-Based and Hybrid Planning}\label{sec:learning_hybrid}

\subsection{Learned Heuristics and Neural Search}\label{subsec:learned_heuristics}

Machine learning techniques have been applied to improve heuristic functions for A* and related algorithms \citep{Zhao2023Survey, PuenteCastro2022}. Neural networks trained on solved instances can learn heuristic functions that estimate goal distance more accurately than hand-crafted geometric heuristics, potentially reducing node expansions \citep{SternLi2021}. However, learned heuristics may violate admissibility, sacrificing the optimality guarantees that are critical for safety-sensitive applications \citep{Russell2021, Das2023}.

Recent work has addressed the admissibility challenge through constrained learning frameworks that enforce $h(n) \leq h^{*}(n)$ during training \citep{Helmert2023}. Hybrid approaches use learned heuristics for initial guidance while falling back to admissible heuristics when optimality certificates are required \citep{Zafar2023}. The computational overhead of neural network evaluation must be weighed against the reduction in node expansions, and current evidence suggests that learned heuristics pay off most on large, complex instances where geometric heuristics provide poor guidance \citep{Zhao2023Survey}.

\subsection{Hybrid Classical-Learning Approaches}\label{subsec:hybrid}

Hybrid approaches marry classical search guarantees with learning-based efficiency \citep{PuenteCastro2022, Zheng2023Risk}. Representative strategies include: using reinforcement learning to select among multiple search algorithms based on problem characteristics \citep{Zhao2023Survey}; training neural networks to predict promising search directions that guide classical algorithms \citep{Das2023}; and employing imitation learning to approximate expert planners while maintaining bounded suboptimality \citep{Choudhury2022}.

In UAV path planning specifically, deep reinforcement learning has been applied to learn navigation policies that implicitly encode risk-aware behaviour \citep{Zheng2023Risk, PuenteCastro2022}. These policies can react to dynamic obstacles without explicit replanning but lack the formal guarantees of classical search methods. Bridging the gap between learned policies and provably optimal planners remains an active research challenge \citep{Zhao2023Survey, SternLi2021}.

\subsection{Deployment Considerations for Learning-Based Methods}\label{subsec:deployment_learning}

Deploying learning-based planning in safety-critical domains raises concerns about generalisation, interpretability, and verification \citep{Raja2024, Wang2024Risk}. Models trained on one environment distribution may perform poorly in novel settings, and the opaque nature of neural decision-making complicates safety certification \citep{Das2023}. For biosecurity applications where autonomous systems must navigate contaminated or restricted areas with high reliability, these concerns currently tilt the balance in favour of classical methods with formal guarantees, supplemented by learning-based components in non-critical subsystems \citep{Phung2021, Tsouros2023}.


\section{Optimisation, Parameter Tuning, and Evaluation Protocols}\label{sec:evaluation}

\subsection{Heuristic and Weight Tuning}\label{subsec:tuning}

The performance of weighted search algorithms is sensitive to the inflation factor $\epsilon$ and, for risk-aware planning, to the risk weight $\lambda$ \citep{Wilt2012, Wang2024Risk}. Small $\epsilon$ values preserve near-optimal solutions but offer limited speedup; large values accelerate search at the expense of substantially suboptimal paths \citep{Russell2021, Kuroiwa2022}. The risk weight $\lambda$ similarly controls the tradeoff between path length and exposure, with appropriate values depending on the risk distribution and mission requirements \citep{Zheng2023Risk, Phung2021}.

Automated parameter tuning through grid search, Bayesian optimisation, or adaptive methods can identify parameter configurations that balance competing objectives \citep{Helmert2023, Das2023}. However, optimal parameters may vary across problem instances, motivating instance-specific or adaptive tuning strategies \citep{Zhao2023Survey}.

\subsection{Evaluation Metrics}\label{subsec:review_metrics}

Standardised evaluation metrics enable meaningful comparison across pathfinding algorithms \citep{Sturtevant2022, Sturtevant2020GPPC}. Commonly reported metrics include:

\begin{itemize}
  \item \textbf{Runtime}: wall-clock computation time, measured under controlled hardware conditions.
  \item \textbf{Nodes expanded}: the number of nodes removed from the open list, reflecting algorithmic efficiency independent of implementation details.
  \item \textbf{Path length}: total cost of the solution path under the specified cost function.
  \item \textbf{Suboptimality ratio}: the ratio of the solution cost to the optimal cost, indicating solution quality.
  \item \textbf{Exposure integral}: for risk-aware planning, the sum of risk values along the path \citep{Wang2024Risk}.
  \item \textbf{Re-plan latency}: the time required to update a solution after an environment change \citep{Andreychuk2022}.
\end{itemize}

\subsection{Benchmark Standards}\label{subsec:benchmarks}

The Moving AI benchmark suite \citep{Sturtevant2022} provides standardised grid maps derived from video games and random generation, supporting reproducible evaluation across research groups. The Grid-Based Path Planning Competition (GPPC) extends this infrastructure with competitive evaluation protocols and time-constrained settings \citep{Sturtevant2020GPPC}. For UAV-specific evaluation, ArduPilot SITL \citep{ArduPilot2024} and Gazebo \citep{GazeboSim2024} provide vehicle-in-the-loop simulation environments that capture dynamics beyond grid-level abstraction \citep{Raja2024, Huang2024}.

Recent benchmarking efforts have highlighted the importance of evaluating algorithms across diverse map types, obstacle densities, and instance difficulties rather than reporting aggregate statistics on a single benchmark \citep{Sturtevant2022, Husvogt2024}. Risk-aware benchmarks remain less standardised, with most studies using custom risk distributions rather than shared benchmark suites \citep{Wang2024Risk, Zheng2023Risk}.


\section{Research Gaps and Challenges}\label{sec:gaps}

The literature review reveals two significant research gaps that motivate the present work. Each gap is aligned with a specific research objective, as summarised in Table~\ref{tab:gap_alignment}.

\subsection{Gap 1: Absence of Corridor-Constrained Search for Risk-Annotated Grids}\label{subsec:gap1}

Jump Point Search and its variants achieve dramatic speedups on uniform-cost grids by exploiting path symmetry to prune redundant node expansions \citep{Hu2021Bidir, Harabor2014}. However, these methods fundamentally require uniform edge costs, ruling out application to risk-annotated grids where cell costs vary continuously \citep{Patel2023, Xu2024Grid}. Subgoal graph methods support weighted grids but require static-environment preprocessing that is invalidated by dynamic updates \citep{Strasser2022, Uras2023}. No existing method provides a corridor-focused search strategy that supports heterogeneous risk-weighted costs while achieving computational reductions comparable to JPS on uniform grids. This gap motivates the development of a corridor-constrained search framework---Incremental Line Search (ILS)---that restricts A* exploration to a narrow band along the direct line of sight, achieving substantial speedups without requiring uniform costs or static preprocessing (Objective~O1).

\subsection{Gap 2: Limited Adaptive Mechanisms for Dynamic Replanning}\label{subsec:gap2}

Existing incremental search methods (D* Lite, LPA*, ARA*) efficiently repair solutions when edge costs change but maintain full-grid data structures and do not constrain the scope of re-exploration \citep{Koenig2002, Koenig2004, Aine2016}. Bounded search approaches use fixed bounds that do not adapt to local environment conditions \citep{Wilt2012, Kuroiwa2022}. No established method provides an adaptive mechanism that dynamically adjusts search scope based on local obstructions and risk concentrations, expanding the search region only where needed while keeping tight bounds elsewhere \citep{Hernandez2021, Andreychuk2022}. This gap motivates the development of an adaptive corridor control mechanism that widens the ILS corridor locally in response to detected obstructions and risk spikes, preserving computational efficiency in unobstructed regions while ensuring path feasibility in complex areas (Objective~O2).

\subsection{Gap--Objective Alignment}\label{subsec:alignment}

Table~\ref{tab:gap_alignment} summarises the alignment between the identified research gaps and the two thesis objectives.

\begin{table}[htb!]
\centering
\caption{Alignment of Research Gaps with Research Objectives}
\label{tab:gap_alignment}
\begin{tabular}{|p{1cm}|p{5.5cm}|p{2cm}|p{4.5cm}|}
\hline
\textbf{Gap} & \textbf{Description} & \textbf{Objective} & \textbf{Proposed Contribution} \\
\hline
G1 & No corridor-constrained search for risk-annotated grids & O1 & Incremental Line Search (ILS) framework \\
\hline
G2 & No adaptive search-scope adjustment for dynamic replanning & O2 & Adaptive corridor control mechanism \\
\hline
\end{tabular}
\end{table}


\section{Summary}\label{sec:review_summary}

This chapter has surveyed the current state of knowledge on grid-based pathfinding for autonomous navigation, spanning classical search algorithms, incremental replanning methods, symmetry reduction techniques, learning-based approaches, and evaluation protocols. The review identifies two research gaps that collectively limit the effectiveness of current approaches for risk-aware autonomous navigation in structured, dynamic environments:

\begin{enumerate}
  \item The absence of corridor-constrained search methods that support heterogeneous risk-weighted costs (Gap~1); and
  \item The lack of adaptive mechanisms that dynamically adjust search scope in response to local obstructions and risk concentrations (Gap~2).
\end{enumerate}

These gaps motivate the two research objectives addressed in the subsequent chapters. Chapter~\ref{chap:methodology} presents the Incremental Line Search framework and the adaptive corridor control mechanism (Objectives~O1 and O2). Chapter~\ref{chap:results} provides the experimental evaluation and discussion.
