\begin{EnAbstract}

Autonomous systems operating in biosecurity-sensitive environments---such as agricultural facilities, healthcare settings, transportation hubs, and border control operations---require efficient, reliable pathfinding algorithms that can compute collision-free routes in real time while accounting for spatially varying risk. This thesis develops, implements, and evaluates novel corridor-based pathfinding techniques for grid maps that substantially reduce computational requirements without sacrificing path quality.

The first contribution is the Incremental Line Search (ILS) framework, which constrains classical search algorithms to a narrow corridor centred on the Bresenham line connecting start and goal positions, expanding the corridor incrementally only when no feasible path exists within the current band. Experiments on 6{,}000 synthetic grids of size $200 \times 200$ at obstacle densities of 10\%, 20\%, and 30\% demonstrate that ILS achieves an average 87.31\% reduction in execution time and 71.44\% reduction in node expansions across five classical algorithms (A*, Dijkstra, BFS, DFS, and Greedy Best-First Search), with all improvements statistically significant ($p < 0.05$). For non-optimal algorithms, ILS additionally improves path quality: DFS path length is reduced by up to 93.74\%.

The second contribution is Adaptive Incremental Line Search (AILS), which extends ILS with per-point density estimation via integral images and a variable-width corridor whose radius at each reference-line cell is a closed-form function of the local obstacle density. Three corridor strategies---Base, Standard, and Predictive---are selected automatically based on density-gradient analysis. Experiments on grids from $50 \times 50$ to $500 \times 500$ with five obstacle topologies demonstrate 51--56\% node reduction on $200 \times 200$ grids (Cohen's $d = 0.76$--$0.82$, $p < 0.001$), scaling to 76.8\% on $500 \times 500$ grids. The Predictive strategy achieves 62.2\% time improvement with 99.8\% optimality rate.

The thesis identifies the complementary strengths of ILS and AILS: ILS excels on spatially uniform environments where a single global corridor width suffices, while AILS provides topological robustness across heterogeneous obstacle configurations by adapting corridor width to local conditions. Both methods are most effective at obstacle densities of 10--25\% on random and open layouts---conditions common in outdoor robotics, warehouse navigation, and agricultural environments.

\textbf{Keywords:} grid-based pathfinding, corridor-constrained search, incremental line search, adaptive corridor, autonomous navigation, biosecurity, risk-aware planning

\end{EnAbstract}
