\begin{EnAbstract}

Autonomous systems that operate in biosecurity-sensitive environments---agricultural facilities, healthcare settings, transportation hubs, border control operations---need pathfinding algorithms that can compute collision-free routes quickly while accounting for spatially varying risk. This thesis develops and evaluates novel corridor-based pathfinding techniques for grid maps that dramatically cut computational costs without giving up path quality.

The first contribution is the Incremental Line Search (ILS) framework. ILS confines classical search algorithms to a narrow corridor centred on the Bresenham line between start and goal, widening the corridor only when no feasible path can be found within the current band. Experiments on 6{,}000 synthetic $200 \times 200$ grids at obstacle densities of 10\%, 20\%, and 30\% show that ILS achieves an average 87.31\% reduction in execution time and 71.44\% reduction in node expansions across five classical algorithms (A*, Dijkstra, BFS, DFS, and Greedy Best-First Search), with all improvements statistically significant ($p < 0.05$). For non-optimal algorithms, ILS also improves path quality---DFS path length drops by up to 93.74\%.

The second contribution is Adaptive Incremental Line Search (AILS), which extends ILS by estimating local obstacle density through integral images and constructing a variable-width corridor whose radius at each reference-line cell depends on how cluttered the surrounding area is. Three corridor strategies---Base, Standard, and Predictive---are chosen automatically based on density-gradient analysis. Experiments on grids ranging from $50 \times 50$ to $500 \times 500$ with five obstacle topologies show 51--56\% node reduction on $200 \times 200$ grids (Cohen's $d = 0.76$--$0.82$, $p < 0.001$), scaling to 76.8\% on $500 \times 500$ grids. The Predictive strategy achieves 62.2\% time improvement with a 99.8\% optimality rate.

ILS and AILS turn out to be complementary. ILS excels on spatially uniform environments where a single global corridor width is enough, while AILS provides robustness across heterogeneous obstacle layouts by adapting the corridor to local conditions. Both methods work best at obstacle densities of 10--25\% on random and open layouts---conditions commonly encountered in outdoor robotics, warehouse navigation, and agricultural environments.

\textbf{Keywords:} grid-based pathfinding, corridor-constrained search, incremental line search, adaptive corridor, autonomous navigation, biosecurity, risk-aware planning

\end{EnAbstract}
