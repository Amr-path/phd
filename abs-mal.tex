\begin{MsAbstract}

Sistem autonomi yang beroperasi dalam persekitaran sensitif biosekuriti---seperti kemudahan pertanian, persekitaran penjagaan kesihatan, hab pengangkutan, dan operasi kawalan sempadan---memerlukan algoritma pencarian laluan yang cekap dan boleh dipercayai untuk mengira laluan bebas perlanggaran dalam masa nyata sambil mengambil kira risiko yang berbeza mengikut ruang. Tesis ini membangunkan, melaksanakan, dan menilai teknik pencarian laluan berasaskan koridor yang baharu untuk peta grid yang mengurangkan keperluan pengiraan dengan ketara tanpa mengorbankan kualiti laluan.

Sumbangan pertama ialah rangka kerja Carian Garis Berperingkat (Incremental Line Search, ILS), yang mengekang algoritma carian klasik kepada koridor sempit yang berpusat pada garis Bresenham yang menghubungkan posisi permulaan dan matlamat, dengan mengembangkan koridor secara berperingkat hanya apabila tiada laluan yang boleh dilaksanakan dalam jalur semasa. Eksperimen ke atas 6{,}000 grid sintetik bersaiz $200 \times 200$ pada ketumpatan halangan 10\%, 20\%, dan 30\% menunjukkan bahawa ILS mencapai purata pengurangan 87.31\% dalam masa pelaksanaan dan 71.44\% pengurangan dalam pengembangan nod merentasi lima algoritma klasik (A*, Dijkstra, BFS, DFS, dan Carian Tamak Terbaik-Dahulu), dengan semua penambahbaikan signifikan secara statistik ($p < 0.05$). Bagi algoritma tidak optimum, ILS turut meningkatkan kualiti laluan: panjang laluan DFS dikurangkan sehingga 93.74\%.

Sumbangan kedua ialah Carian Garis Berperingkat Adaptif (Adaptive Incremental Line Search, AILS), yang memperluaskan ILS dengan anggaran ketumpatan per-titik melalui imej kamiran dan koridor lebar berubah yang jejarinya pada setiap sel garis rujukan merupakan fungsi bentuk tertutup bagi ketumpatan halangan setempat. Tiga strategi koridor---Asas, Standard, dan Ramalan---dipilih secara automatik berdasarkan analisis kecerunan ketumpatan. Eksperimen ke atas grid dari $50 \times 50$ hingga $500 \times 500$ dengan lima topologi halangan menunjukkan 51--56\% pengurangan nod pada grid $200 \times 200$ (Cohen's $d = 0.76$--$0.82$, $p < 0.001$), meningkat kepada 76.8\% pada grid $500 \times 500$. Strategi Ramalan mencapai 62.2\% penambahbaikan masa dengan kadar keoptimuman 99.8\%.

Tesis ini mengenal pasti kekuatan pelengkap ILS dan AILS: ILS unggul dalam persekitaran ketumpatan seragam di mana satu lebar koridor global mencukupi, manakala AILS memberikan keteguhan topologi merentasi konfigurasi halangan heterogen dengan menyesuaikan lebar koridor kepada keadaan setempat. Kedua-dua kaedah paling berkesan pada ketumpatan halangan 10--25\% pada susun atur rawak dan terbuka---keadaan yang lazim dalam robotik luar, navigasi gudang, dan persekitaran pertanian.

\textbf{Kata Kunci:} pencarian laluan berasaskan grid, carian terkekang koridor, carian garis berperingkat, koridor adaptif, navigasi autonomi, biosekuriti, perancangan sedar risiko

\end{MsAbstract}
